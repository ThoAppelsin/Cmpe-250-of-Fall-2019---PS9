% 1610, 149, 54, 43 and 32.
\documentclass[aspectratio=32]{beamer}

\usetheme{Flip}
\usepackage{mathtools}
\usepackage{listings}

\lstset{
	basicstyle=\fontseries{c}\selectfont\footnotesize,
	columns=fullflexible,
	language=C++,
	numbers=left,
	tabsize=4,
	showspaces=false,
	showstringspaces=false,
	stepnumber=1,
	numbersep=10pt,
	commentstyle=\fontseries{lc}\selectfont\itshape
}

% Thanks to: https://tex.stackexchange.com/a/85578/69346
\lstMakeShortInline{!}

\usepackage{tikz}
\usetikzlibrary{positioning,calc}
\tikzset{onslide/.code args={<#1>#2}{%
  \only<#1>{\pgfkeysalso{#2}} % \pgfkeysalso doesn't change the path
}}

\makeatletter
\newenvironment<>{btHighlight}[1][]
{\begin{onlyenv}#2\begingroup\tikzset{bt@Highlight@par/.style={#1}}\begin{lrbox}{\@tempboxa}}
{\end{lrbox}\bt@HL@box[bt@Highlight@par]{\@tempboxa}\endgroup\end{onlyenv}}

\newcommand<>\btHL[1][]{%
  \only#2{\begin{btHighlight}[#1]\bgroup\aftergroup\bt@HL@endenv}%
}
\def\bt@HL@endenv{%
  \end{btHighlight}%   
  \egroup
}
\newcommand{\bt@HL@box}[2][]{%
  \tikz[#1]{%
    \pgfpathrectangle{\pgfpoint{1pt}{0pt}}{\pgfpoint{\wd #2}{\ht #2}}%
    \pgfusepath{use as bounding box}%
    \node[anchor=base west, fill=orange!30,outer sep=0pt,inner xsep=1pt, inner ysep=0pt, rounded corners=3pt, minimum height=\ht\strutbox+1pt,#1]{\raisebox{1pt}{\strut}\strut\usebox{#2}};
  }%
}
\makeatother

%%% End thanks

\title[Disjoint Sets]{
	Disjoint Sets \\
	A visualization
}
\author[Utkan Gezer]{M. Utkan Gezer}

\begin{document}

\begin{frame}[containsverbatim]
\frametitle{The data structure}
\begin{center}
\begin{tabular}{c}
\begin{lstlisting}
class DJSets { vector<int> s; };

DJSets::DJSets( int n ) : s( n, -1 ) { }

void DJSets::unionSets( int rA, int rB ) {
    if( s[ rB ] < s[ rA ] ) s[ rA ] = rB;
    else {
        if( s[ rA ] == s[ rB ] ) --s[ rA ];
        s[ rB ] = rA;
    }
}

int DJSets::find( int x ) {
    if( s[ x ] < 0 )	return x;
    else				return find( s[ x ] );
}
\end{lstlisting}
\end{tabular}
\end{center}
\end{frame}


\begin{frame}[fragile]
\frametitle{The visualization}
\begin{columns}

\begin{column}{5cm}
	\begin{block}{\texttt{main}}
		\setlength\textheight{1em}
		\footnotesize
		\ttfamily
		\only<1>{// BEGIN\\}
		\only<1-2>{DJSets djSet\{10\};\\}
		\only<1>{cout << djSet.find(4);\\}
		\only<2-3>{cout << djSet.find(4); // 4\\}
		\only<2>{cout << djSet.find(2);\\}
		\only<3-6>{cout << djSet.find(2); // 2\\}
		\only<3-7>{djSet.unionSets(4, 2);\\}
		\only<4-6>{cout << djSet.find(4);\\}
		\only<7-9>{cout << djSet.find(4); // 4\\}
		\only<7>{cout << djSet.find(2);\\}
		\only<8>{cout << djSet.find(2); // f(4)\\}
		\only<9-10>{cout << djSet.find(2); // 4\\}
		\only<8-12>{djSet.unionSets(4, 3);\\}
		\only<10-11>{cout << djSet.find(3); // f(4)\\}
		\only<12>{cout << djSet.find(3); // 4\\}
		\only<11-12>{// END}
	\end{block}

	\begin{block}{\texttt{vector s}}
		\centering
		\begin{tikzpicture}[scale=0.49]
			\fill<5> [black] (4,0) -|++ (1,1) -| cycle;
			\fill<6> [black] (2,0) -|++ (1,1) -| cycle;
			\fill<10> [black] (3,0) -|++ (1,1) -| cycle;

			\draw[step=1] (0,0) grid (10,1);
			\node at (.5,.5) {-1};
			\node at (1.5,.5) {-1};
			\node at (2.5,.5) {\only<-5>{-1}\only<6->{4}};
			\node at (3.5,.5) {\only<-9>{-1}\only<10->{4}};
			\node at (4.5,.5) {\only<-4>{-1}\only<5->{-2}};
			\node at (5.5,.5) {-1};
			\node at (6.5,.5) {-1};
			\node at (7.5,.5) {-1};
			\node at (8.5,.5) {-1};
			\node at (9.5,.5) {-1};
		\end{tikzpicture}
	\end{block}

	\begin{block}{The graph}
		\centering
		\small
		\begin{tikzpicture}[scale=0.5]
			\fill<6,8>[black] (n2) circle (4mm);
			\fill<10,11>[black] (n3) circle (4mm);
			\foreach \i in {0,...,9} {
				\node[draw, circle, inner sep=2] (n\i) at (\i*36:1.5) {\clap{\i}};
			}
			\path[draw, ->] (n0) --++ (0*36:1.0);
			\path[draw, ->] (n1) --++ (1*36:1.0);
			\path<-5>[draw, ->] (n2) --++ (2*36:1.0);
			\path<-9>[draw, ->] (n3) --++ (3*36:1.0);
			\path[draw, ->] (n4) --++ (4*36:1.0);
			\path[draw, ->] (n5) --++ (5*36:1.0);
			\path[draw, ->] (n6) --++ (6*36:1.0);
			\path[draw, ->] (n7) --++ (7*36:1.0);
			\path[draw, ->] (n8) --++ (8*36:1.0);
			\path[draw, ->] (n9) --++ (9*36:1.0);

			\path<7,9->[draw, ->] (n2) to [bend left] (n4);
			\path<6,8>[draw, ->, black, very thick] (n2) to [bend left] (n4);

			\path<12->[draw, ->] (n3) to[out=110, in=90, looseness=2] (n4);
			\path<10,11>[draw, ->, black, very thick] (n3) to[out=110, in=90, looseness=2] (n4);
		\end{tikzpicture}
	\end{block}
\end{column}

\begin{column}{5cm}
\begin{lstlisting}[
	moredelim={**[is][{\btHL<1>}]{@1}{@}},
	moredelim={**[is][{\btHL<2-3,7,9,12>}]{@2}{@}},
	moredelim={**[is][{\btHL<4>}]{@3}{@}},
	moredelim={**[is][{\btHL<5>}]{@4}{@}},
	moredelim={**[is][{\btHL<6,10>}]{@5}{@}},
	moredelim={**[is][{\btHL<8,11>}]{@6}{@}}
]
class DJSets { vector<int> s; };

@1DJSets::DJSets( int n ) : s( n, -1 ) { }@

void DJSets::unionSets( int rA, int rB ) {
	// assume: rA, rB roots and different
    if( s[ rB ] < s[ rA ] ) s[ rA ] = rB;
    @3else {@
        @4if( s[ rA ] == s[ rB ] ) --s[ rA ];@
        @5s[ rB ] = rA;@
    }
}

int DJSets::find( int x ) {
    @2if( s[ x ] < 0 )	return x;@
    @6else				return find( s[ x ] );@
}
\end{lstlisting}
\end{column}

\end{columns}
\end{frame}

\end{document}
